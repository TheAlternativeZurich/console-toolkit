\documentclass{TheAlternativeCourse}

\usepackage{menukeys}

\title{Console Toolkit Exercises}
\date{}

\begin{document}
\maketitle

\section{Introduction}

Welcome to this exercise session! This is a practical exercise on the Linux
command line.  If you have any question about the exercises, feel free to ask
the helpers for assistance.


\subsection*{Difficulty}

The exercises are designed in such a way that you will have to find solutions
by reading the manuals and using search engines. You are expected to come up
with appropriate solutions on your own, there will be no step-by-step guided
exercises!

\subsection*{Liability}

By taking part in this exercise, you acknowledge that you alone are responsible
for your computer. TheAlternative and its parent organisation, the Student
Sustainability Commission, will not be held liable for any damages or loss of
data.

\pagebreak

\section{Basics}

To solve some of the exercises, you will need to download some files. These
can be found in a Git repository. If you don't have \texttt{git} installed,
now is the time to do so. \\
%
\begin{exercisebox}{Getting the repository}
    Open a terminal and enter the following: \\
    \texttt{git clone INSERT\_HERE}
\end{exercisebox}

\subsection{Getting help}

\texttt{man} is a command you will need throughout all of the following
exercises.  It stands for ''manual`` and shows you what almost any given
command can do and which options are available for it.  Almost all commands
provide such a ''manpage``, which is typically written by the developers
themselves. \\
%
\begin{exercisebox}{The manual's manual}
    The \texttt{man} command itself has its own manpage. Type \texttt{man man}
    to access it, use the arrow keys to scroll, and press \keys{q} to exit.
    Have a look around, especially at the different section numbers!
\end{exercisebox}
%
An alternative to using the man pages is to try and see if a command has a help
option. This is usually displayed by appending the option \texttt{--help} after the
name of the command, which will often display much shorter and concise instructions
on how to use a command. \\
%
\begin{exercisebox}{The help option}
    Look at the help output of \texttt{ls}. What is the flag to see hidden files?
\end{exercisebox}

\section{Files}

This section introduces you to basic commands related to files. We will first look at a command that can be used for creating empty files.\\

The \texttt{touch} command is most frequently used for creating new, empty files. If you check the manual, you will see that it can also change certain file timestamps (access and modification times).\\
%
\begin{exercisebox}{The touch command}
	Switch to your home directory. Type \texttt{touch file}. A new file should have been created. Confirm that this is the case.
\end{exercisebox}\\
%
We already saw \texttt{cat} in the previous section. It is time to introduce another important command to look at files: \texttt{less}. This command is great if you want to look at large bodies of text because, unlike \texttt{cat}, it has a built-in scroll function. Once you are done looking at the file, you can quit with \texttt{q}.\\
%
\begin{exercisebox}{Less is more}
	There is also a \texttt{more} command. It is older than \texttt{less} and thus less frequently used. Find a large file (or create onw) and look at it with \texttt{less FILE}. How does it differ from \texttt{more}? Read the manuals if you're confused.
\end{exercisebox}

\subsection{Permissions}

We will now look at two commands to fiddle with permissions.\\

\texttt{chown} stands for "change owner". It is used to change the owner of a given file. It can also change the group ownership.

\begin{exercisebox}{Changing ownership}
	Create a new, empty file (hint: \texttt{touch}) and use \texttt{chown} to change its ownership to the \texttt{root} user (you will probably need \texttt{sudo} for this). Try deleting it. Then change the ownership back to your own user.
\end{exercisebox}

A very similar command is \texttt{chmod}. \texttt{chmod} stands for "change file mode bits" and controls the following permissions on any given file:
\begin{itemize}
	\item Read: Who can see the file data.
	\item Write: Who can modify the file.
	\item Execute: Who is allowed to run the file (like a program or a script)
\end{itemize}

\begin{hintbox}
Permissions can be read using `ls -l`.
\end{hintbox}

\end{document}
