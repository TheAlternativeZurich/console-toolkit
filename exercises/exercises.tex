\documentclass{TheAlternativeCourse}

\usepackage{menukeys}
\usepackage{tabularx}
\usepackage{float}
\usepackage{caption}

\title{Console Toolkit Exercises}
\date{}

\begin{document}
\maketitle

\section{Introduction}

Welcome to this exercise session! This is a practical exercise on the Linux
command line.  If you have any question about the exercises, feel free to ask
the helpers for assistance.


\subsection*{Difficulty}

The exercises are designed in such a way that you will have to find solutions
by reading the manuals and using search engines. You are expected to come up
with appropriate solutions on your own, there will be no step-by-step guided
exercises! \\\\
%
There will be quite simple exercises for complete beginners at the start,
but there are also some much more difficult ones further into the exercise.
Feel free to skip ahead!

\subsection*{Liability}

By taking part in this exercise, you acknowledge that you alone are responsible
for your computer. TheAlternative and its parent organisation, the Student
Sustainability Commission, will not be held liable for any damages or loss of
data.

\pagebreak

\section{Basics}

To solve some of the exercises, you will need to download some files. These
can be found in a Git repository. If you don't have \texttt{git} installed,
now is the time to do so. \\
%
\begin{exercisebox}{Getting the repository}
    Open a terminal and enter the following: \\
    \texttt{git clone INSERT\_HERE}
\end{exercisebox}

\subsection{Running a command}

To run a command, you just type it into the console and hit enter to run it.
Most useful commands require some arguments. Arguments are given after the name of
the command, separated by spaces. There is also so-called \emph{flags}, which
set some options to alter the behaviour of a command. They are of the form
\texttt{-x} (single letter form) or \texttt{--long-option} (long format). \\\\
%
An example of a more involved command:

\begin{cmdbox}
\texttt{ls -l --human-readable \textasciitilde/Downloads}
\end{cmdbox}

\subsection{Getting help}

\texttt{man} is a command you will need throughout the following
exercises.  It stands for ''manual`` and shows you what almost any given
command can do and which options are available for it.  Almost all commands
provide such a ''manpage``, which is typically written by the developers
themselves. \\
%
\begin{exercisebox}{The manual's manual}
    The \texttt{man} command itself has its own manpage. Type \texttt{man man}
    to access it, use the arrow keys to scroll, and press \keys{q} to exit.
    Have a look around, especially at the different section numbers.
    What is the section number for system calls?
\end{exercisebox}
%
An alternative to using the man pages is to try and see if a command has a help
option. This is usually displayed by appending the option \texttt{--help} after
the name of the command, which will often display much shorter and concise
instructions
on how to use a command. \\
%
\begin{exercisebox}{The help option}
    Look at the help output of \texttt{ls}. What is the flag to see hidden files?
\end{exercisebox}

\subsection{Navigating directories}
%
You can navigate your file system in the terminal, just like you could with
a graphical program. Below are the most important commands.
%
\begin{table}[H]
    \centering
    \rowcolors{1}{lightblue}{white}
    \begin{tcolorbox}[%
        enhanced,
        width=1.0\linewidth,
        fuzzy shadow={1mm}{-1mm}{0mm}{0.1mm}{black!50!white},
        tabularx={>{\centering\arraybackslash}l|>{\centering\arraybackslash}X},
        title={Commands for navigating directories}]
        \textbf{Command} & \textbf{Description} \\
        \texttt{pwd} & Display the current working directory. \\
        \texttt{tree \$dir} & Get a visualization of the directory tree under
            the current working directory. \\
        \texttt{ls \$dir} & List all files and directories in the current
            working directory. \\
        \texttt{cd \$dir} & Change the current working directory to the given
            directory. \\
    \end{tcolorbox}
    \label{tab1}
\end{table}
%
\begin{exercisebox}{Making yourself at home}
    Have a look around your home directory and try out the mentioned commands.
    You will most likely recognise the layout from your graphical file browser.
    Remember you can always return to your home directory by running
    \texttt{cd} without any arguments.
\end{exercisebox}
%
\subsection{Modifying directories}
%
Just like in graphical programs, you can create and delete directories on the
command line.
%
\begin{table}[H]
    \centering
    \rowcolors{1}{lightblue}{white}
    \begin{tcolorbox}[%
        enhanced,
        fuzzy shadow={1mm}{-1mm}{0mm}{0.1mm}{black!50!white},
        width=1.0\linewidth,
        tabularx={>{\centering\arraybackslash}l|>{\centering\arraybackslash}X},
        title={Commands for modifying directories}]
        \textbf{Command} & \textbf{Description} \\
        \texttt{mkdir \$dir} & Create a new directory with the given name. \\
        \texttt{rmdir \$dir} & Remove a directory. Will not work with
            non-empty directories. \\
        \texttt{rm -r \$dir} &  Remove a directory and its contents.
            {\color{red}Attention! There is no trashcan on the command line!
            Files and directories will be deleted irrevocably.} \\
        \texttt{cp -r \$source \$target} &  Copy a source directory
            (recursively with all its contents) to a target. \\
        \texttt{mv \$source \$target} & Move a file or directory.
            Also used to rename files/directories. \\
    \end{tcolorbox}
    \label{tab2}
\end{table}
%
\begin{exercisebox}{Creating your first directory}
    Create a directory for the following exercises (you can call it
    \texttt{console-toolkit}).  Change into that directory and create a file
    call \texttt{notes}. Then move the exercise file directory you downloaded
    in task 2.1 into your newly created directory.
\end{exercisebox}
%
\subsection{Viewing files}
If you want to view the contents of a text file, you have the following
options:
%
\begin{table}[H]
    \centering
    \rowcolors{1}{lightblue}{white}
    \begin{tcolorbox}[%
        enhanced,
        fuzzy shadow={1mm}{-1mm}{0mm}{0.1mm}{black!50!white},
        width=1.0\linewidth,
        tabularx={>{\centering\arraybackslash}l|>{\centering\arraybackslash}X},
        title={Commands for viewing files}]
        \textbf{Command} & \textbf{Description} \\
        \texttt{cat \$file} & Output a file to the terminal. \\
        \texttt{head \$file} & Output the first couple of lines to the terminal. \\
        \texttt{tail \$file} & Output the last couple of lines to the terminal. \\
        \texttt{less \$file} & Browse a file in a visual viewer. \\
        \texttt{touch \$file} & Create a new file when it doesn't exist yet. \\
    \end{tcolorbox}
    \label{tab3}
\end{table}
%
\begin{exercisebox}{Log files}
    \texttt{tail} is often used to inspect errors log files. Use tail to find the
    last three lines of \texttt{dmesg\_log} (in the exercise files).
\end{exercisebox}

\subsection{Console tips \& tricks}

There are many little things that can make your life using the terminal a
little bit easier.  Following is a table with the most important keyboard
shortcuts you can use to speed up your workflow.

%
\begin{table}[H]
    \centering
    \rowcolors{1}{lightblue}{white}
    \begin{tcolorbox}[%
        enhanced,
        fuzzy shadow={1mm}{-1mm}{0mm}{0.1mm}{black!50!white},
        width=1.0\linewidth,
        tabularx={>{\centering\arraybackslash}l|>{\centering\arraybackslash}X},
        title={Commands for viewing files}]
        \keys{\ctrl}+\keys{w} & Delete one word backwards. \\
        \keys{\ctrl}+\keys{u} & Delete the entire line. \\
        \keys{\ctrl}+\keys{l} & Clear the terminal. \\
        \keys{\ctrl}+\keys{a} & Go to the beginning of the line. \\
        \keys{\ctrl}+\keys{e} & Go to the end of the line. \\
        \keys{\ctrl}+\keys{c} & Terminate the currently runnig process. \\
        \keys{\ctrl}+\keys{d} & Quit the shell. \\
    \end{tcolorbox}
    \label{tab1}
\end{table}

\section{Files}

This section introduces you to basic commands related to files. We will first look at a command that can be used for creating empty files.\\

The \texttt{touch} command is most frequently used for creating new, empty files. If you check the manual, you will see that it can also change certain file timestamps (access and modification times).\\
%
\begin{exercisebox}{The magic touch}
	Switch to your home directory. Type \texttt{touch file}. A new file should have been created. Confirm that this is the case.
\end{exercisebox}\\
%
We already saw \texttt{cat} in the previous section. It is time to introduce another important command to look at files: \texttt{less}. This command is great if you want to look at large bodies of text because, unlike \texttt{cat}, it has a built-in scroll function. Once you are done looking at the file, you can quit with \keys{q}.\\
%
\begin{exercisebox}{Less is more}
	There is also a \texttt{more} command. It is older than \texttt{less} and thus less frequently used. Find a large file (or create onw) and look at it with \texttt{less FILE}. How does it differ from \texttt{more}? Read the manuals if you're confused.
\end{exercisebox}

\subsection{Permissions}

We will now look at two commands to fiddle with permissions.\\

\texttt{chown} stands for "change owner". It is used to change the owner of a given file. It can also change the group ownership.

\begin{exercisebox}{Get off my property}
	Create a new, empty file (hint: \texttt{touch}) and use \texttt{chown} to change its ownership to the \texttt{root} user (you will probably need \texttt{sudo} for this). Try deleting it. Then change the ownership back to your own user.
\end{exercisebox}

A very similar command is \texttt{chmod}. \texttt{chmod} stands for "change file mode bits" and controls the following permissions on any given file:
\begin{itemize}
	\item Read: Who can see the file data.
	\item Write: Who can modify the file.
	\item Execute: Who is allowed to run the file (like a program or a script)
\end{itemize}

\hintbox{Permissions can be read using \texttt{ls -l}.}

\section{Remote machines}

This section will be about working with remote machines (servers). You will
connect to one and move files from and to it. \\

%
% pterodactyl@vsos.ethz.ch
% hackinsession,BoredHacker

\end{document}
